The data used for this study has been recorded during the Run-II of the LHC. It corresponds to the 2015-2018 data taking period, the total integrated luminosity corresponds to 139.2 $\ifb	$ of proton-proton collisions at a centre-of-mass energy of $\sqrt{s}=13$ TeV.

Monte Carlo (MC) samples are used for signal and background process simulation. For $\Zll$ events two MC samples are used, $\POWPY[8]$ and $\SHERPA$. For the background processes we use $\POWPY[8]$, except for the di-boson samples which are simulated with $\SHERPA$. Table \ref{Table3} shows the MC generators used for each process.

\begin{table}[htbp]
	\centering
	\begin{tabular}{cc}
		\hline
		\multicolumn{1}{|c|}{Process}  & \multicolumn{1}{c|}{Event Generator} \\ \hline
		$Z\to\tau\tau$                 & $\POWPY[8]$ and $\SHERPA$           \\
		$Z\to ee$                      & $\POWPY[8]$ and $\SHERPA$           \\
		$Z\to\mu\mu$                   & $\POWPY[8]$ and $\SHERPA$           \\
		$W\to l\nu_l$				   & $\POWPY[8]$                       \\
		$t\bar{t}$                     & $\POWPY[8]$                       \\
		Single $t$                     & $\POWPY[8]$                       \\
		Diboson                        & $\SHERPA$                       \\ \hline
	\end{tabular}
	\caption{List of MC event generators used.}
	\label{Table3}
\end{table}