%-------------------------------------------------------------------------------
% This file contains the title, author and abstract.
% It also contains all relevant document numbers used for an ATLAS note.
%-------------------------------------------------------------------------------

% Title
\AtlasTitle{Tau-ID Scale Factors for high-$\pt$ taus using boosted $Z\to\tau\tau$ events.}

% Draft version:
% Should be 1.0 for the first circulation, and 2.0 for the second circulation.
% If given, adds draft version on front page, a 'DRAFT' box on top of each other page, 
% and line numbers.
% Comment or remove in final version.
\AtlasVersion{0.1}

% Abstract - % directly after { is important for correct indentation
\AtlasAbstract{%
  Different algorithms have been trained by the ATLAS collaboration to separate true semi-hadronically decaying taus from QCD jets. The efficiency of these algorithms is compared in simulation and data and correction factors are derived to account for the differences that may arise from the simulation limitations. This study aims to evaluate the correction factors for the tau-ID algorithms for high-$\pt$ taus. Highly boosted $Z\to\tauhad\taulep$ events are used in this study. Two final states are explored, where $\taulep\to e,\mu+\nu 's$.  The scale factors are calculated as a double ratio between $Z\to\tauhad\taulep$ and $\Zll$ events. This method, aims to cancel phenomenological, luminosity and light-lepton related uncertainties. In this report, we present the value obtained for the scale factors for the classifier \textit{tight-ID} working point. The values for 1-prong taus with $\pt>45$ GeV are $\text{SF}_{\text{Tight-ID}}=0.964\pm 0.02\text{(stat)}\pm 0.024\text{(lumi)}\pm 0.028\text{(sys)}$ for the muon-tau final state and $\text{SF}_{\text{Tight-ID}}=0.968\pm 0.023\text{(stat)}\pm 0.023\text{(lumi)}\pm 0.029\text{(sys)}$ for the electron-tau final state. In the case of 3-prong taus, the values are are $\text{SF}_{\text{Tight-ID}}=0.854\pm 0.039\text{(stat)}\pm 0.024\text{(lumi)}\pm 0.028\text{(sys)}$ and $\text{SF}_{\text{Tight-ID}}=0.923\pm 0.053\text{(stat)}\pm 0.023\text{(lumi)}\pm 0.029\text{(sys)}$ respectively.
}

% Author - this does not work with revtex (add it after \begin{document})
\author{The ATLAS Collaboration}

% Authors and list of contributors to the analysis
% \AtlasAuthorContributor also adds the name to the author list
% Include package latex/atlascontribute to use this
% Use authblk package if there are multiple authors, which is included by latex/atlascontribute
% \usepackage{authblk}
% Use the following 3 lines to have all institutes on one line
% \makeatletter
% \renewcommand\AB@affilsepx{, \protect\Affilfont}
% \makeatother
% \renewcommand\Authands{, } % avoid ``. and'' for last author
% \renewcommand\Affilfont{\itshape\small} % affiliation formatting
% \AtlasAuthorContributor{First AtlasAuthorContributor}{a}{Author's contribution.}
% \AtlasAuthorContributor{Second AtlasAuthorContributor}{b}{Author's contribution.}
% \AtlasAuthorContributor{Third AtlasAuthorContributor}{a}{Author's contribution.}
% \AtlasContributor{Fourth AtlasContributor}{Contribution to the analysis.}
 \author[a]{Diego Baron}
 \author[a]{Terry Wyatt}
% \author[b]{Third Author}
 \affil[a]{University of Manchester}
% \affil[b]{Another Institution}

% If a special author list should be indicated via a link use the following code:
% Include the two lines below if you do not use atlasstyle:
% \usepackage[marginal,hang]{footmisc}
% \setlength{\footnotemargin}{0.5em}
% Use the following lines in all cases:
% \usepackage{authblk}
% \author{The ATLAS Collaboration%
% \thanks{The full author list can be found at:\newline
%   \url{https://atlas.web.cern.ch/Atlas/PUBNOTES/ATL-PHYS-PUB-2020-007/authorlist.pdf}}
% }

% ATLAS reference code, to help ATLAS members to locate the paper
\AtlasRefCode{GROUP-2020-XX}

% ATLAS note number. Can be an COM, INT, PUB or CONF note
% \AtlasNote{ATLAS-CONF-2020-XXX}
% \AtlasNote{ATL-PHYS-PUB-2020-XXX}
% \AtlasNote{ATL-COM-PHYS-2020-XXX}
