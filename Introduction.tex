The tau lepton has a proper decay length of 87 $\mu$m \cite{PhysRevD.98.030001}. For this reason, tau leptons usually decay before they can reach the innermost layer of the ATLAS detector. Therefore, only the decay products of the tau leptons can be observed. The tau lepton mass of 1.777 GeV makes this particle the only lepton that can decay into hadrons \cite{PhysRevD.98.030001}. Tau decays can be either leptonically ($\tau\to\nu_\tau\nu_l l$, $l=e,\mu$) or semi-hadronically ($\tau\to\nu_\tau+$hadrons), the latter ones are commonly referred simply as hadronic tau decays ($\tauhad$). Muons and electrons from leptonic tau decays do not have enough kinematical features that can make them easily distinguishable from prompt muons or electrons. In the case of $\tauhad$ decays, which represent approximately 65\% of the tau branching fraction, the decay modes include one or three charged pions. Therefore, the signature of these decays are jets with one or three charged tracks, with a charge correlation and being more collimated than jets initiated from quark or gluon radiation.

Algorithms trained to identify true $\tauhad$ and separating them from misidentified $\tauhad$  candidates, originating from QCD processes, have been developed \cite{Deutsch:2680523}. The most novel of these algorithms is a recurrent neural network (RNN) that uses track and calorimeter information in order to classify true and fake $\tauhad$ candidates. The RNN is now used as the default tau identification criteria for the data recorded by the ATLAS experiment from 2015 to 2018 and has been used as well in 2018 tau lepton triggers.

When the efficiency of the RNN algorithm is measured in data and simulation, a correction or scale factor (SF) is derived and then applied to the simulation in order for the signal efficiency to agree between data and simulation. For high-$\pt$ taus, $t\bar{t}$ events are often used to derive the SFs, since the high top quark mass boosts the decays. However, by measuring the SFs in this way one could be correcting for possible new physics coming from lepton universality violation effects in the W boson decays.

This work is aimed to measure the correction factors for tau identification in the high-$\pt$ regime using boosted $Z\to\tauhad\taulep$ events. Since lepton universality has been tested to far better precision in Z decays, our measurement is safer from lepton universality violating effects.

This report is structured as follows. In Chapter \ref{DataMC}, we present the description of the data and simulation samples used in this work. Chapter \ref{Ana}  describes the analysis methodology, the event selection is presented and the definition of the scale factors is given. The background estimation is discussed in Chapter \ref{BG}. In Chapter \ref{Sys}, the systematic uncertainties for our analysis are presented. The results of our analysis and a discussion is in Chapter \ref{Results}. Finally in Chapter \ref{Conclusion}, conclusions and prospects are discussed.
