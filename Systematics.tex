Due to the fact that our SFs are defined as a double ratio, we expect some of the systematic uncertainties to cancel at least at first order. Therefore, luminosity, light-lepton and physics modelling uncertainties are expected to be reduced. Unfortunately, tau-lepton reconstruction and MJ background estimation uncertainties do not cancel in the double ratio method. Then, we expect them to dominate the measurement precision.

After calculating the integrated luminosity uncertainty for the SFs, we find that it is less than $0.07\%$, so we just ignore it in our final results. The systematic uncertainties used to report the SFs are presented in Table \ref{Tab5}. These numbers have been provided by Terry Wyatt and Sam Dysch, based on their previous experience working on an analysis that makes use of $Z\to\tauhad\taulep$ events. A full estimate of the systematic uncertainties is in progress.
\begin{table}[htbp]
	\centering
	\begin{tabular}{cc}
		\hline
		\multicolumn{1}{|c|}{Source}        & \multicolumn{1}{c|}{Sys. Uncertainty (\%)} \\ \hline
		Electron ID efficiency              & 0.8                                        \\
		Muon ID Efficiency                  & 0.2                                        \\
		Electron $\pt$ scale and resolution & 0.4                                        \\
		Muon $\pt$ scale and resolution     & 0.3                                        \\
		Tau $\pt$ scale                     & 1.9                                        \\
		Electron trigger efficiency         & 0.1                                        \\
		Muon trigger efficiency             & 0.4                                        \\ 
	\end{tabular}
	\caption{Systematic uncertainties used in this study. At the time of writing this report we have not directly estimated all of our systematic uncertainties. However, we expect that some of them will get reduced by the double ratio method.}
	\label{Tab5}
\end{table}

