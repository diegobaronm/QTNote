The study presented is a work in progress. This report in particular is submitted as my 2nd-year report.

Given that the jets produced in hadronic tau decays have particular features, the ATLAS collaboration has developed algorithms to help with the identification of these decays. Simulation correction factors account for the difference in the efficiency of these algorithms between real data and simulation. The aim of our study, is to derive the correction factors using highly-boosted $Z\to\tauhad\taulep$ events. A preliminary result of the value of the correction factor for the tight-ID working point is presented in this work. The values obtained for $SF_{\text{Tight-ID}}$ in the $e\tauhad$ and $\mu\tauhad$ final states are consistent within the uncorrelated uncertainties, for both 1-prong and 3-prong taus.


The generators used for our study tend to underestimate the Z$\pt$ in the high momentum region and give different predictions. To correct for this effect we have used the double ratio between $\Zll$ ($l=e,\mu$) and $Z\to\tauhad\taulep$ events. This offers us the possibility to derive the corrections factors for both generators and also claim a systematic uncertainty accounting for the Z$\pt$ modelling. Our results show that in fact, there is agreement between the different generator configurations. Confirming that the double ratio method help us to reduce the uncertainty coming from the Z$\pt$ modelling. As an additional feature of this method, we expect that at least at first order there will be a cancellation of the light-lepton related uncertainties. In particular the remaining luminosity uncertainty of this method is negligible for our measurement.


As plans for the future, we could extend our study to the looser tau-ID working points. In this region the challenge will be to control the contribution coming from the MJ background. 

Finally, we hope that all we have learnt about the highly boosted di-tau systems can be applied to a future analysis.  The goal of this study will be to observe vector boson fusion production in final states with tau leptons. In this case the di-tau systems will recoil against the jets produced in the event. More information about previous observations of this topic can be find in \cite{Aad:2014dta,Aaboud:2017emo}. The cited measurement is done with $\Zll$ decays and our measurement will focus in the $Z\to\tauhad\taulep$ process.