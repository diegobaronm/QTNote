This study was presented as Diego Baron ATLAS qualification task.

Given that the jets produced in hadronic tau decays have particular features, the ATLAS collaboration has developed algorithms to help with the identification of these decays. Simulation correction factors account for the difference in the efficiency of these algorithms between real data and simulation. The main goal of this study, is to derive the correction factors using highly-boosted $Z\to\tauhad\taulep$ events. The values of the correction factors for the tight-ID working point are presented in this work. 

The generators used for this study tend to underestimate the Z$\pt$ in the high momentum region and give different predictions. To correct for this effect we have used the double ratio between $\Zll$ ($l=e,\mu$) and $Z\to\tauhad\taulep$ events. This offers the possibility to derive the corrections factors for both generators and achieve a small systematic uncertainty for the Z$\pt$ modelling. The results show that there is agreement between the Z-$\pt$ re-weighted and the standard samples, confirming that the double ratio method reduces the uncertainty coming from the Z$\pt$ modelling. As an additional feature of this method, some experimental uncertainties cancel partially at first order. In particular, the remaining luminosity uncertainty in this method is negligible for the measurement.


As plans for the future, one could try to extend this study to the looser tau-ID working points. In those regions, the challenge will be to calculate a reliable estimate of the MJ background. This measurement can also be used to compare the description of the tau ID efficiency between different MC generator versions as they become available. One advantage of this method is that since we are not taking ratios of efficiencies itself, an independent measurement of tau-ID can be combined with this one to obtain a fist measurement of tau reconstruction SFs.

Finally, we hope that all we have learnt about the highly boosted di-tau systems can be applied to a future analysis.  The goal of this study will be to observe vector boson fusion production in final states with tau leptons. In this case the di-tau systems will recoil against the jets produced in the event. More information about previous observations of this topic can be find in \cite{Aad:2014dta,Aaboud:2017emo}. The cited measurement is done with $\Zll$ decays and the measurement will focus in the $Z\to\tauhad\taulep$ process.